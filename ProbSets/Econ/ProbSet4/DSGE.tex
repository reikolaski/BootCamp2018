\documentclass[letterpaper,12pt]{article}
\usepackage{array}
\usepackage{threeparttable}
\usepackage{geometry}
\geometry{letterpaper,tmargin=1in,bmargin=1in,lmargin=0.75in,rmargin=0.75in}
\usepackage{fancyhdr,lastpage}
\pagestyle{fancy}
\lhead{}
\chead{}
\rhead{}
\lfoot{}
\cfoot{}
\rfoot{\footnotesize\textsl{Page \thepage\ of \pageref{LastPage}}}
\renewcommand\headrulewidth{0pt}
\renewcommand\footrulewidth{0pt}
\usepackage[format=hang,font=normalsize,labelfont=bf]{caption}
\usepackage{listings}
\lstset{frame=single,
  language=Python,
  showstringspaces=false,
  columns=flexible,
  basicstyle={\small\ttfamily},
  numbers=none,
  breaklines=true,
  breakatwhitespace=true
  tabsize=3
}
\usepackage{amsmath}
\usepackage{amssymb}
\usepackage{amsthm}
\usepackage{harvard}
\usepackage{setspace}
\usepackage{float,color}
\usepackage[pdftex]{graphicx}
\usepackage{hyperref}
\usepackage{mathrsfs}
\let\vec\mathbf
\hypersetup{colorlinks,linkcolor=red,urlcolor=blue}
\theoremstyle{definition}
\newtheorem{theorem}{Theorem}
\newtheorem{acknowledgement}[theorem]{Acknowledgement}
\newtheorem{algorithm}[theorem]{Algorithm}
\newtheorem{axiom}[theorem]{Axiom}
\newtheorem{case}[theorem]{Case}
\newtheorem{claim}[theorem]{Claim}
\newtheorem{conclusion}[theorem]{Conclusion}
\newtheorem{condition}[theorem]{Condition}
\newtheorem{conjecture}[theorem]{Conjecture}
\newtheorem{corollary}[theorem]{Corollary}
\newtheorem{criterion}[theorem]{Criterion}
\newtheorem{definition}[theorem]{Definition}
\newtheorem{derivation}{Derivation} % Number derivations on their own
\newtheorem{example}[theorem]{Example}
\newtheorem{exercise}[theorem]{Exercise}
\newtheorem{lemma}[theorem]{Lemma}
\newtheorem{notation}[theorem]{Notation}
\newtheorem{problem}[theorem]{Problem}
\newtheorem{proposition}{Proposition} % Number propositions on their own
\newtheorem{remark}[theorem]{Remark}
\newtheorem{solution}[theorem]{Solution}
\newtheorem{summary}[theorem]{Summary}
\bibliographystyle{aer}
\newcommand\ve{\varepsilon}
\newcommand\boldline{\arrayrulewidth{1pt}\hline}

\setlength{\parindent}{0pt}

\begin{document}

\begin{flushleft}
  \textbf{\large{DSGE Problem Set}} \\
  Reiko Laski
\end{flushleft}

\textbf{\large{DSGE}}

\textbf{Exercise 1}
\begin{align*}
  K_{t+1} &= Ae^{z_t}K_t^\alpha \\
  K_{t+2} &= Ae^{z_{t+1}}K_{t+1}^\alpha \\
  &= Ae^{z_{t+1}}(Ae^{z_t}K_t^\alpha)^\alpha
\end{align*}

\begin{align*}
  \frac{1}{e^{z_t}K_t^\alpha - Ae^{z_t}K_t^\alpha}
  &=
  \beta E_t\Big\{\frac{\alpha e^{z_{t+1}}(Ae^{z_t}K_t^\alpha)^{\alpha - 1}}{e^{z_{t+1}}(Ae^{z_t}K_t^\alpha)^\alpha-Ae^{z_{t+1}}(Ae^{z_t}K_t^\alpha)^\alpha}\Big\}
  \\
  \frac{1}{e^{z_t}K_t^\alpha(1-A)}
  &=
  \beta E_t\Big\{\frac{\alpha}{Ae^{z_t}K_t^\alpha(1-A)}\Big\}
  \\
  A &= \alpha\beta
\end{align*}

\textbf{Exercise 2}
\begin{align*}
  &c_t = (1-\tau) [w_t\ell_t+(r_t-\delta)k_t]+k_t+T_t-k_{t+1}
  \\
  &c_t^{-1} = \beta E_t\{c_{t+1}^{-1}[(r_{t+1}-\delta)(1-\tau)+1]\}
  \\
  &\frac{a}{1-\ell_t} = c_t^{-1}w_t(1-\tau)
  \\
  &r_t = \alpha e^{z_t}k_t^{\alpha-1} l_t^{1-\alpha}
  \\
  &w_t = (1-\alpha)e^{z_t}k_t^{\alpha}\ell_t^{-\alpha}
  \\
  &\tau[w_t\ell_t+(r_t-\delta)k_t] = T_t
  \\
  &z_t = (1-\rho_z)\bar{z}+\rho_zz_{t-1}+\epsilon_t^z; \ \ \epsilon_t^z \sim i.i.d.(0, \sigma_z^2)
\end{align*}

\textbf{Exercise 3}
\begin{align*}
  &c_t = (1-\tau) [w_t\ell_t+(r_t-\delta)k_t]+k_t+T_t-k_{t+1}
  \\
  &c_t^{-\gamma} = \beta E_t\{c_{t+1}^{-\gamma}[(r_{t+1}-\delta)(1-\tau)+1]\}
  \\
  &\frac{a}{1-\ell_t} = c_t^{-\gamma}w_t(1-\tau)
  \\
  &r_t = \alpha e^{z_t}k_t^{\alpha-1} \ell_t^{1-\alpha}
  \\
  &w_t = (1-\alpha)e^{z_t}k_t^{\alpha}\ell_t^{-\alpha}
  \\
  &\tau[w_t\ell_t+(r_t-\delta)k_t] = T_t
  \\
  &z_t = (1-\rho_z)\bar{z}+\rho_zz_{t-1}+\epsilon_t^z; \ \ \epsilon_t^z \sim i.i.d.(0, \sigma_z^2)
\end{align*}

\textbf{Exercise 4}
\begin{align*}
  &c_t = (1-\tau) [w_t\ell_t+(r_t-\delta)k_t]+k_t+T_t-k_{t+1}
  \\
  &c_t^{-\gamma} = \beta E_t\{c_{t+1}^{-\gamma}[(r_{t+1}-\delta)(1-\tau)+1]\}
  \\
  &a(1-\ell_t)^{-\xi} = c_t^{-\gamma}w_t(1-\tau)
  \\
  &r_t = \frac{\alpha e_{z_t} k_t^{\eta-1}}{[\alpha k_t^\eta+(1-\alpha)\ell_t^\eta]^{1/\eta}}
  \\
  &w_t = \frac{(1-\alpha) e_{z_t} \ell_t^{\eta-1}}{[\alpha k_t^\eta+(1-\alpha)\ell_t^\eta]^{1/\eta}}
  \\
  &\tau[w_t\ell_t+(r_t-\delta)k_t] = T_t
  \\
  &z_t = (1-\rho_z)\bar{z}+\rho_zz_{t-1}+\epsilon_t^z; \ \ \epsilon_t^z \sim i.i.d.(0, \sigma_z^2)
\end{align*}

\textbf{Exercise 5} \\
Characterizing equations:
\begin{align*}
  &c_t = (1-\tau) [w_t+(r_t-\delta)k_t]+k_t+T_t-k_{t+1}
  \\
  &c_t^{-\gamma} = \beta E_t\{c_{t+1}^{-\gamma}[(r_{t+1}-\delta)(1-\tau)+1]\}
  \\
  &0 = c_t^{-\gamma}w_t(1-\tau)
  \\
  &r_t = \alpha k_t^{\alpha-1} (e^{z_t})^{1-\alpha}
  \\
  &w_t = (1-\alpha)k_t^{\alpha}(e^{z_t})^{1-\alpha}
  \\
  &\tau[w_t+(r_t-\delta)k_t] = T_t
  \\
  &z_t = (1-\rho_z)\bar{z}+\rho_zz_{t-1}+\epsilon_t^z; \ \ \epsilon_t^z \sim i.i.d.(0, \sigma_z^2)
\end{align*}
Steady state:
\begin{align*}
  &\bar{c} = (1-\tau) [\bar{w}+(\bar{r}-\delta)\bar{k}]+\bar{T}
  \\
  &1 = \beta E_t[(\bar{r}-\delta)(1-\tau)+1]
  \\
  &0 = \bar{c}^{-\gamma}\bar{w}(1-\tau)
  \\
  &\bar{r} = \alpha \bar{k}^{\alpha-1} (e^{\bar{z}})^{1-\alpha}
  \\
  &\bar{w} = (1-\alpha)\bar{k}^{\alpha}(e^{\bar{z}})^{1-\alpha}
  \\
  &\tau[\bar{w}+(\bar{r}-\delta)\bar{k}] = \bar{T}
\end{align*}

Solve for the steady state value of $k$:
\begin{align*}
  &1 = \beta[(\bar{r}-\delta)(1-\tau)+1]
  \\
  &1 = \beta[(\alpha \bar{k}^{\alpha-1}(e^{\bar{z}})^{1-\alpha}-\delta)(1-\tau)+1]
  \\
  &1 = \beta(\alpha \bar{k}^{\alpha-1}(e^{\bar{z}})^{1-\alpha}-\delta)(1-\tau)+\beta
  \\
  &\frac{1-\beta}{\beta(1-\tau)} = \alpha \bar{k}^{\alpha-1}(e^{\bar{z}})^{1-\alpha}-\delta
  \\
  &\bar{k} = \Big(\frac{1}{\alpha(e^{\bar{z}})^{1-\alpha}} \Big[\frac{1-\beta}{\beta(1-\tau)}+\delta \Big]\Big)^{\frac{1}{\alpha-1}}
\end{align*}

\textbf{Exercise 6} \\
Characterizing equations:
\begin{align*}
  &c_t = (1-\tau) [w_t\ell_t+(r_t-\delta)k_t]+k_t+T_t-k_{t+1}
  \\
  &c_t^{-\gamma} = \beta E_t\{c_{t+1}^{-\gamma}[(r_{t+1}-\delta)(1-\tau)+1]\}
  \\
  &a(1-\ell_t)^{-\xi} = c_t^{-\gamma}w_t(1-\tau)
  \\
  &r_t = \alpha k_t^{\alpha-1} (\ell_t e^{z_t})^{1-\alpha}
  \\
  &w_t = (1-\alpha)k_t^{\alpha}(e^{z_t})^{1-\alpha}\ell_t^{-\alpha}
  \\
  &\tau[w_t\ell_t+(r_t-\delta)k_t] = T_t
  \\
  &z_t = (1-\rho_z)\bar{z}+\rho_zz_{t-1}+\epsilon_t^z; \ \ \epsilon_t^z \sim i.i.d.(0, \sigma_z^2)
\end{align*}
Steady state:
\begin{align*}
  &\bar{c} = (1-\tau) [\bar{w}\bar{\ell}+(\bar{r}-\delta)\bar{k}]+\bar{T}
  \\
  &1 = \beta E_t[(\bar{r}-\delta)(1-\tau)+1]
  \\
  &a(1-\bar{\ell})^{-\xi} = \bar{c}^{-\gamma}\bar{w}(1-\tau)
  \\
  &\bar{r} = \alpha \bar{k}^{\alpha-1} (\bar{\ell} e^{\bar{z}})^{1-\alpha}
  \\
  &\bar{w} = (1-\alpha)\bar{k}^{\alpha}(e^{\bar{z}})^{1-\alpha}\bar{\ell}^{-\alpha}
  \\
  &\tau[\bar{w}\bar{\ell}+(\bar{r}-\delta)\bar{k}] = \bar{T}
\end{align*}

\textbf{\large{Linearization}}

\textbf{Exercise 3}
\begin{align*}
  &E_t\{FX_{t+1}+G{X_t}+HX_{t-1}+LZ_{t+1}+MZ_t\}=0
  \\
  &E_t\{F(PX_t+QZ_{t+1})+G(PX_{t-1}+QZ_t)+HX_{t-1}+L(NZ_t+\epsilon_t)+MZ_t\}=0
  \\
  &E_t\{FPX_t+FQZ_{t+1}+GPX_{t-1}+GQZ_t+HX_{t-1}+LNZ_t+L\epsilon_t+MZ_t\}=0
  \\
  &E_t\{FP(PX_{t-1}+QZ_t)+FQ(NZ_t+\epsilon_t)+GPX_{t-1}+GQZ_t+HX_{t-1}+LNZ_t+L\epsilon_t+MZ_t\}=0
  \\
  &E_t\{FP^2X_{t-1}+FPQZ_t+FQNZ_t+FQ\epsilon_t+GPX_{t-1}+GQZ_t+HX_{t-1}+LNZ_t+L\epsilon_t+MZ_t\}=0
  \\
  &FP^2X_{t-1}+FPQZ_t+FQNZ_t+GPX_{t-1}+GQZ_t+HX_{t-1}+LNZ_t+MZ_t=0
  \\
  &(FP^2+GP+H)X_{t-1}+(FPQ+FQN+FQ+LN+M)Z_t=0
  \\
  &[(FP+G)P+H]X_{t-1}+[(FQ+L)N+(FP+G)Q+M]Z_t=0
\end{align*}

\vspace*{\fill}

Sorry that I didn't finish this problem set. I was really sick this week, and I decided to prioritize health and sleep. Thank you for three wonderful lectures, and again, I apologize for my unfinished work. \quad -Reiko

\end{document}
