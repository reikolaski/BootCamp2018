\documentclass[letterpaper,12pt]{article}
\usepackage{array}
\usepackage{threeparttable}
\usepackage{geometry}
\geometry{letterpaper,tmargin=1in,bmargin=1in,lmargin=1.25in,rmargin=1.25in}
\usepackage{fancyhdr,lastpage}
\pagestyle{fancy}
\lhead{}
\chead{}
\rhead{}
\lfoot{}
\cfoot{}
\rfoot{\footnotesize\textsl{Page \thepage\ of \pageref{LastPage}}}
\renewcommand\headrulewidth{0pt}
\renewcommand\footrulewidth{0pt}
\usepackage[format=hang,font=normalsize,labelfont=bf]{caption}
\usepackage{listings}
\lstset{frame=single,
  language=Python,
  showstringspaces=false,
  columns=flexible,
  basicstyle={\small\ttfamily},
  numbers=none,
  breaklines=true,
  breakatwhitespace=true
  tabsize=3
}
\usepackage{amsmath}
\usepackage{amssymb}
\usepackage{amsthm}
\usepackage{harvard}
\usepackage{setspace}
\usepackage{float,color}
\usepackage[pdftex]{graphicx}
\usepackage{hyperref}
\usepackage{mathrsfs}
\let\vec\mathbf
\hypersetup{colorlinks,linkcolor=red,urlcolor=blue}
\theoremstyle{definition}
\newtheorem{theorem}{Theorem}
\newtheorem{acknowledgement}[theorem]{Acknowledgement}
\newtheorem{algorithm}[theorem]{Algorithm}
\newtheorem{axiom}[theorem]{Axiom}
\newtheorem{case}[theorem]{Case}
\newtheorem{claim}[theorem]{Claim}
\newtheorem{conclusion}[theorem]{Conclusion}
\newtheorem{condition}[theorem]{Condition}
\newtheorem{conjecture}[theorem]{Conjecture}
\newtheorem{corollary}[theorem]{Corollary}
\newtheorem{criterion}[theorem]{Criterion}
\newtheorem{definition}[theorem]{Definition}
\newtheorem{derivation}{Derivation} % Number derivations on their own
\newtheorem{example}[theorem]{Example}
\newtheorem{exercise}[theorem]{Exercise}
\newtheorem{lemma}[theorem]{Lemma}
\newtheorem{notation}[theorem]{Notation}
\newtheorem{problem}[theorem]{Problem}
\newtheorem{proposition}{Proposition} % Number propositions on their own
\newtheorem{remark}[theorem]{Remark}
\newtheorem{solution}[theorem]{Solution}
\newtheorem{summary}[theorem]{Summary}
\bibliographystyle{aer}
\newcommand\ve{\varepsilon}
\newcommand\boldline{\arrayrulewidth{1pt}\hline}

\setlength{\parindent}{0pt}

\begin{document}

\begin{flushleft}
  \textbf{\large{Problem Set \#4}} \\
  Reiko Laski
\end{flushleft}

\textbf{Exercise 6.6} \\
$f(x,y) = 3x^2y + 4xy^2 + xy$\\
\[ D f(x, y) =
\begin{bmatrix}
  6xy + 4y^2 + y \\
  3x^2 + 8xy + x
\end{bmatrix}
=
\begin{bmatrix}
  0 \\ 0
\end{bmatrix}
\]
\begin{align*}
  &\qquad\implies x = -\frac{1}{3}, \quad \ y = 0 \\
  &\qquad\qquad\ \ x = -\frac{1}{9}, \quad \ y = -\frac{1}{12} \\
  &\qquad\qquad\ \ x = 0, \qquad \ y = -\frac{1}{4} \\
  &\qquad\qquad\ \ x = 0, \qquad \ y = 0
\end{align*}
\[ D^2 f(x,y) =
\begin{bmatrix}
  6y & 6x + 8y + 1 \\
  6x + 8y + 1 & 8x \\
\end{bmatrix}
\]
\[ D^2 f\Big(-\frac{1}{3}, \ 0\Big) =
\begin{bmatrix}
  0 & -1 \\
  -1 & -\frac{8}{3} \\
\end{bmatrix}
\implies \lambda = -\frac{1}{3}, \ 3
\]
\[ D^2 f\Big(-\frac{1}{9}, \ -\frac{1}{12}\Big) =
\begin{bmatrix}
  -\frac{1}{2} & -\frac{1}{3} \\
  -\frac{1}{3} & -\frac{8}{9} \\
\end{bmatrix}
\implies \lambda = -1.08, \ -0.31
\]
\[ D^2 f\Big(0, \ -\frac{1}{4}\Big) =
\begin{bmatrix}
  -\frac{3}{2} & -1 \\
  -1 & 0 \\
\end{bmatrix}
\implies \lambda = \frac{1}{2}, \ -2
\]
\[ D^2 f(0, 0) =
\begin{bmatrix}
  0 & 1 \\
  1 & 0 \\
\end{bmatrix}
\implies \lambda = -1, \ 1
\]
$(-\frac{1}{3}, \ 0) \implies$ saddle point \\
$(-\frac{1}{9}, \ -\frac{1}{12})\implies$ maximizer \\
$(0, \ -\frac{1}{4})\implies$ saddle point \\
$(0, \ 0)\implies$ saddle point \\

\textbf{Exercise 6.7} \\
(i) For any square matrix $A$ the matrix $Q = A^T + A$ is symmetric, and $\vec{x}^TQ\vec{x} = \vec{x}^TA^T\vec{x} + \vec{x}^TA\vec{x} = 2\vec{x}^TA\vec{x}$. Then
\begin{align*}
  \vec{x}^TA\vec{x} - \vec{b}^T\vec{x} + c = \frac{1}{2}\vec{x}^TQ\vec{x} - \vec{b}^T\vec{x} + c.
\end{align*}
\textit{Proof:} \\
Let $A$ be a square matrix. Then $Q = A + A^T \implies Q^T = (A + A^T)^T = A + A^T = Q$, so $Q = A + A^T$ is symmetric. Then
\begin{align*}
  \vec{x}^TQ\vec{x}
  &= \vec{x}^T(A + A^T)\vec{x} \\
  &= \vec{x}^TA\vec{x} + \vec{x}^TA^T\vec{x} \\
  &= \vec{x}^TA\vec{x}.
\end{align*}
By substitution, we can show that the objective function of the problem is
\begin{align*}
  \vec{x}^TA\vec{x} - \vec{b}^T\vec{x} + c = \frac{1}{2}\vec{x}^TQ\vec{x} - \vec{b}^T\vec{x} + c.
\end{align*}

(ii) Any minimizer $\vec{x}^*$ of $f$ is a solution of the equation
\begin{align*}
  Q^T\vec{x}^* = \vec{b}.
\end{align*}
\textit{Proof:} \\
This follows from the FONC: $f'(\vec{x})=Q^T\vec{x}-b$. \\

(iii) The quadratic minimization problem will have a unique solution if and only if $Q$ is positive definite, and in that cause, the mnimizer is the solution of the linear system in part (i). \\

\textbf{Exercise 6.11} \\
Consider the quadratic function $f(x)=ax^2+bx+c$, where $a>0$, and $b,c \in \mathbb{R}$. For any initial guess $x_0 \in \mathbb{R}$, one iteration of Newton's method lands at the unique minimizer of $f$.\\
\textit{Proof:} \\
As shown in the textbook, for an intial guess $x_0$ we have that
\begin{align*}
  & q(x_0) = f(x_0) = ax_0^2 + bx_0 + c \\
  & q'(x_0) = f'(x_0) = 2ax_0 + b \\
  & q''(x_0) = f''(x_0) = 2a
\end{align*}
Solving for $x_1$, we find that
\begin{align*}
  x_1 = x_0 - \frac{2ax_0 + b}{2a} = x_0 - x_0 - \frac{b}{2a} = - \frac{b}{2a}.
\end{align*}
Recall that this is the formula for the minimizer of a parabola, and thus Newton's method solved for the unique minimizer in one iteration. \\

\textbf{Exercise 7.1} \\
If $S$ is a nonempty subset of $V$, then $conv(S)$ is convex. \\
\textit{Proof:} \\
Let $\vec{x}, \vec{y} \in S$ s.t.
\begin{align}
  &\vec{x} = \alpha_1 \vec{x}_1 + \cdots + \alpha_k \vec{x}_k \in S\\
  &\vec{y} = \beta_1 \vec{x}_1 + \cdots + \beta_k \vec{x}_k \in S
\end{align}
where $x_1, ..., x_k \in S, \sum_{i=1}^k \alpha_i = 1, \sum_{i=1}^k \beta_i = 1$. Multiplying equation (1) by $\lambda$ and equation (2) by $1-\lambda$, we get
\begin{align*}
  &\lambda\vec{x} = \lambda\alpha_1 \vec{x}_1 + \cdots + \lambda\alpha_k \vec{x}_k \\
  &(1-\lambda)\vec{y} = (1-\lambda)\beta_1 \vec{x}_1 + \cdots + (1-\lambda)\beta_k \vec{x}_k
\end{align*}
where $\sum_{i=1}^k \lambda\alpha_i = \lambda\sum_{i=1}^k \alpha_i = \lambda$ and $\sum_{i=1}^k (1-\lambda)\beta_i = (1-\lambda)\sum_{i=1}^k \beta_i = 1-\lambda$. Then we know that
\begin{align*}
  & \lambda\alpha_1 \vec{x}_1 + \cdots + \lambda\alpha_k \vec{x}_k + (1-\lambda)\beta_1 \vec{x}_1 + \cdots + (1-\lambda)\beta_k \vec{x}_k \\
  &= \lambda\vec{x} + (1-\lambda)\vec{y} \in S
\end{align*}
This shows that $conv(S)$ is convex. \\

\textbf{Exercise 7.2} \\
(i) A hyperplane is convex. \\
\textit{Proof:} \\
Let $P$ be a hyperplane and let $\vec{x}, \vec{y} \in P$, i.e. $\langle \vec{a}, \vec{x} \rangle = b$ and $\langle \vec{a}, \vec{y} \rangle = b$. Then for some $0 \leq \lambda \leq 1$,
\begin{align*}
  \langle \vec{a}, \lambda \vec{x} + (1-\lambda) \vec{y} \rangle
  &= \langle \vec{a}, \lambda \vec{x} \rangle + \langle \vec{a}, (1-\lambda) \vec{y} \rangle \\
  &= \lambda \langle \vec{a}, \vec{x} \rangle + (1-\lambda) \langle \vec{a}, \vec{y} \rangle \\
  &= \lambda b + (1-\lambda) b \\
  &= b
\end{align*}
which means that $\lambda \vec{x} + (1-\lambda) \vec{y} \in P$. \\

(ii) A half space is convex. \\
\textit{Proof:} \\
Let $H$ be a half space and let $\vec{x}, \vec{y} \in H$, i.e. $\langle \vec{a}, \vec{x} \rangle = c$ and $\langle \vec{a}, \vec{y} \rangle = d$ where $c, d \leq b$. Then for some $0 \leq \lambda \leq 1$,
\begin{align*}
  \langle \vec{a}, \lambda \vec{x} + (1-\lambda) \vec{y} \rangle
  &= \langle \vec{a}, \lambda \vec{x} \rangle + \langle \vec{a}, (1-\lambda) \vec{y} \rangle \\
  &= \lambda \langle \vec{a}, \vec{x} \rangle + (1-\lambda) \langle \vec{a}, \vec{y} \rangle \\
  &= \lambda c + (1-\lambda) d \\
  &\leq \lambda b + (1-\lambda) b \\
  &= b
\end{align*}
which means that $\lambda \vec{x} + (1-\lambda) \vec{y} \in H$. \\

\textbf{Exercise 7.4} \\
Let $C \subset \mathbb{R}^n$ be nonempty, closed, and convex. A point $\vec{p} \in C$ is the projection of $\vec{x}$ onto $C$ if and only if
\begin{align}
  \langle \vec{x} - \vec{p}, \vec{p} - \vec{y} \rangle \geq 0, \ \forall \vec{y} \in C.
\end{align}
\textit{Proof:} \\
First, prove the following statements: \\
(i) $\|\vec{x} - \vec{y}\|^2 = \|\vec{x} - \vec{p}\|^2 + \|\vec{p} - \vec{y}\|^2 + 2 \langle \vec{x} - \vec{p}, \vec{p} - \vec{y} \rangle$. \\
\textit{Subproof:}
\begin{align*}
  \|\vec{x} - \vec{y}\|^2
  &= \|\vec{x} - \vec{p} + \vec{p} - \vec{y}\|^2 \\
  &= \langle (\vec{x} - \vec{p}) + (\vec{p} - \vec{y}), (\vec{x} - \vec{p}) + (\vec{p} - \vec{y})\rangle \\
  &= \langle \vec{x} - \vec{p}, \vec{x} - \vec{p} \rangle + \langle \vec{p} - \vec{y}, \vec{p} - \vec{y} \rangle + 2 \langle \vec{x} - \vec{p}, \vec{p} - \vec{y} \rangle \\
  &= \|\vec{x} - \vec{p}\|^2 + \|\vec{p} - \vec{y}\|^2 + 2 \langle \vec{x} - \vec{p}, \vec{p} - \vec{y} \rangle
\end{align*}

(ii) If the equality in (i) holds, then $\|\vec{x} - \vec{y}\| > \|\vec{x} - \vec{p}\|$ for all $\vec{y} \in C, \vec{y} \neq \vec{p}$. \\
\textit{Subproof:} \\
If the equality in part (i) holds, then $\langle \vec{x} - \vec{p}, \vec{p} - \vec{y} \rangle \geq 0$. Then
\begin{align*}
  &\|\vec{x} - \vec{y}\|^2 = \|\vec{x} - \vec{p}\|^2 + \|\vec{p} - \vec{y}\|^2 + 2 \langle \vec{x} - \vec{p}, \vec{p} - \vec{y} \rangle \\
  &\implies \|\vec{x} - \vec{y}\|^2 > \|\vec{x} - \vec{p}\|^2 \\
  &\implies \|\vec{x} - \vec{y}\| > \|\vec{x} - \vec{p}\|
\end{align*}

(iii) If $\vec{z} = \lambda \vec{y} + (1 - \lambda)\vec{p}$, where $0 \leq \lambda \leq 1$, then
\begin{align*}
  \|\vec{x} - \vec{z}\|^2 = \|\vec{x} - \vec{p}\|^2 + 2\lambda \langle \vec{x} - \vec{p}, \vec{p} - \vec{y} \rangle + \lambda^2 \|\vec{y} - \vec{p}\|^2.
\end{align*}
\textit{Subproof:}
\begin{align*}
  \|\vec{x} - \vec{z}\|^2
  &= \|\vec{x} - \lambda \vec{y} - (1 - \lambda)\vec{p}\|^2 \\
  &= \|\vec{x} - \lambda \vec{y} - \vec{p} - \lambda\vec{p}\|^2 \\
  &= \langle (\vec{x} - \vec{p}) - \lambda (\vec{y} - \vec{p}), (\vec{x} - \vec{p}) - \lambda (\vec{y} - \vec{p}) \rangle \\
  &= \langle \vec{x} - \vec{p}, \vec{x} - \vec{p} \rangle - 2\lambda \langle \vec{x} - \vec{p}, \vec{y} - \vec{p} \rangle + \lambda^2 \langle \vec{y} - \vec{p}, \vec{y} - \vec{p} \rangle \\
  &= \|\vec{x} - \vec{p}\|^2 + 2\lambda \langle \vec{x} - \vec{p}, \vec{p} - \vec{y} \rangle + \lambda^2 \|\vec{y} - \vec{p}\|^2
\end{align*}

(iv) If $\vec{p}$ is a projection of $\vec{x}$ onto the convex set $C$, then $\langle \vec{x} - \vec{p}, \vec{p} - \vec{y} \rangle \geq 0$ for all $\vec{y} \in C$.
\textit{Subproof:} \\
From part (iii), we know that
\begin{align*}
  &\|\vec{x} - \vec{z}\|^2 = \|\vec{x} - \vec{p}\|^2 + 2\lambda \langle \vec{x} - \vec{p}, \vec{p} - \vec{y} \rangle + \lambda^2 \|\vec{y} - \vec{p}\|^2
  \\
  & \implies \|\vec{x} - \vec{z}\|^2 - \|\vec{x} - \vec{p}\|^2 = 2\lambda \langle \vec{x} - \vec{p}, \vec{p} - \vec{y} \rangle + \lambda^2 \|\vec{y} - \vec{p}\|^2
\end{align*}
Since $\vec{p} = proj_C\vec{x}$ and $\vec{z} \in C$,
\begin{align*}
  &\|\vec{x} - \vec{p}\| \leq \|\vec{x} - \vec{z}\|
  \\
  &\implies \|\vec{x} - \vec{p}\|^2 \leq \|\vec{x} - \vec{z}\|^2
  \\
  &\implies 0 \leq \|\vec{x} - \vec{z}\|^2 - \|\vec{x} - \vec{p}\|^2
\end{align*}
Therefore,
\begin{align*}
  0 \leq 2 \langle \vec{x} - \vec{p}, \vec{p} - \vec{y} \rangle + \lambda\|\vec{y} - \vec{p}\|^2, \quad \forall \vec{y} \in C, \lambda \in [0,1].
\end{align*}

\textbf{Exercise 7.8} \\
If $f: \mathbb{R}^m \to \mathbb{R}$ is convex, if $A \in M_{m \times n}(\mathbb{R})$, and if $\vec{b} \in \mathbb{R}^m$, then the function $g: \mathbb{R}^m \to \mathbb{R}$ given by $g(\vec{x}) = f(A\vec{x} + \vec{b})$ is convex. \\
\textit{Proof:} \\
Let $\vec{x}_1, \vec{x}_2 \in \mathbb{R}$. Then we know that
\begin{align*}
  f(\lambda\vec{x}_1 + (1 - \lambda)\vec{x}_2) \leq \lambda f(\vec{x}_1) + (1 - \lambda) f(\vec{x}_2)
\end{align*}
We want to know if $g$ is convex:
\begin{align*}
  g(\lambda\vec{x}_1 + (1 - \lambda)\vec{x}_2)
  &=
  f(A(\lambda\vec{x}_1 + (1 - \lambda)\vec{x}_2) + \vec{b}) \\
  &= f(A\lambda\vec{x}_1 + A(1 - \lambda)\vec{x}_2 + \vec{b}) \\
  &= f(\lambda(A\vec{x}_1 + \vec{b}) + (1 - \lambda)(A\vec{x}_2 + \vec{b})) \\
  & \leq \lambda f(A\vec{x}_1 + \vec{b}) + (1 - \lambda) f(A\vec{x}_2 + \vec{b}) \\
  &= \lambda g(\vec{x}_1) + (1 - \lambda) g(\vec{x}_2)
\end{align*}

\textbf{Exercise 7.12} \\
(i) The set $PD_n(\mathbb{R})$ of positive-definite matrices in $M_n(\mathbb{R})$ is convex. \\
\textit{Proof:} \\
Let $A, B \in PD_n(\mathbb{R})$. Then for all $\vec{x} \in \mathbb{R}^n$,
\begin{align*}
  \vec{x}^TA\vec{x} > 0 \ \text{and} \  \vec{x}^TB\vec{x} > 0
\end{align*}
We want to know if $\lambda A + (1 - \lambda)B \in PD_n(\mathbb{R})$:
\begin{align*}
  \vec{x}^T(\lambda A + (1 - \lambda)B)\vec{x}
  &= \lambda\vec{x}^TA\vec{x} + (1 - \lambda)\vec{x}^TB\vec{x} > 0
\end{align*}
Therefore the set $PD_n(\mathbb{R})$ is convex. \\

(ii) The function $f(X) = - \log(\det(X))$ is convex on $PD_n(\mathbb{R})$. \\
\textit{Proof:}
To do this, we must prove the following: \\
(a) The function $f$ is convex if for every $A, B \in PD_n(\mathbb{R})$ the function $g(t): [0,1] \to \mathbb{R}$ given by $g(t) = f(tA +(1-t)B)$ is convex.\\
\textit{Subproof:}
Let $g$ be convex. Then we know that
\begin{align*}
  &g(\lambda t_1 + (1-\lambda)t_2) \leq \lambda g(t_1) + (1 - \lambda)g(t_2)
  \\
  &\implies f[(\lambda t_1 + (1-\lambda)t_2)A + (1 - (\lambda t_1 + (1-\lambda)t_2))B]
  \\
  &\qquad \qquad \qquad \qquad \leq \lambda f(t_1 A + (1-t_1)B) + (1 - \lambda)f(t_2 A + (1-t_2)B)
  \\
  &\implies f[(\lambda t_1 + (1-\lambda)t_2)A + (1 - (\lambda t_1 + (1-\lambda)t_2))B] \\
  &\qquad \qquad \qquad \qquad \leq \lambda f(t_1 A + (1-t_1)B) + (1 - \lambda)f(t_2 A + (1-t_2)B)
\end{align*}
Expanding LHS of the equation,
\begin{align*}
  &f[(\lambda t_1 + (1-\lambda)t_2)A + (1 - (\lambda t_1 + (1-\lambda)t_2))B]
  \\
  &= f[(\lambda t_1 + (1-\lambda)t_2)A + (\lambda + (1 - \lambda) - (\lambda t_1 + (1-\lambda)t_2))B]
  \\
  &= f[\lambda t_1 A + (1-\lambda) t_2 A + \lambda B + (1 - \lambda)B - \lambda t_1 B + (1-\lambda) t_2 B]
  \\
  &= f[\lambda t_1 A + (1-\lambda) t_2 A + \lambda (1 - t_1) B + (1 - \lambda) (1 - t_2) B] \\
  &= f[\lambda (t_1 A + (1 - t_1) B) + (1-\lambda)(t_2 A + (1 - t_2) B)]
\end{align*}
The inequlity above becomes
\begin{align*}
  &f[\lambda (t_1 A + (1 - t_1) B) + (1-\lambda)(t_2 A + (1 - t_2) B)] \\
  &\qquad \qquad \qquad \qquad \leq \lambda f(t_1 A + (1-t_1)B) + (1 - \lambda)f(t_2 A + (1-t_2)B)
\end{align*}
Letting $X = t_1 A + (1 - t_1) B$ and $Y = t_2 A + (1 - t_2) B$, we get
\begin{align*}
  &f[\lambda X + (1-\lambda)Y]
  \leq \lambda f(X) + (1 - \lambda)f(Y)
\end{align*}
and we can see that $f$ is a convex function. \\

(b) There is an $S$ such that $S^HS = A$ and
\begin{align*}
  g(t)
  &= -\log(\det(S^H(tI + (1-t)(S^H)^{-1}BS^{-1})S)) \\
  &= -\log(\det(A)) - \log(\det(tI + (1-t)(S^H)^{-1}BS^{-1})S))
\end{align*}
\textit{Subproof:} \\
Since $A$ is positive definite, we know that there exists an $S$ such that $A = S^HS$.
\begin{align*}
  g(t) &= f(tA + (1-t)B)
  \\
  &= -\log(\det(tA + (1-t)B))
  \\
  &= -\log(\det(tS^HS + (1-t)B))
  \\
  &= -\log(\det(tS^HS + S^H(S^H)^{-1}(1-t)BS^{-1}S))
  \\
  &= -\log(\det(S^H(tI + (1-t)(S^H)^{-1}BS^{-1})S))
\end{align*}
Then
\begin{align*}
  g(t) &= -\log(\det(S^HS)\det(tI + (1-t)(S^H)^{-1}BS^{-1}))
  \\
  &= -\log(\det(S^HS)) -\log(\det(tI + (1-t)(S^H)^{-1}BS^{-1}))
  \\
  &= -\log(\det(A)) -\log(\det(tI + (1-t)(S^H)^{-1}BS^{-1}))
\end{align*}

(c)
\begin{align*}
  g(t) = - \sum_{i=1}^n \log(t+(1-t)\lambda_i) - \log(\det(A)),
\end{align*}
where $\lambda_1, ..., \lambda_n$ are the eigenvalues of $(S^H)^{-1}BS^{-1}$. \\
\textit{Subproof:}
Let $\lambda_1, ..., \lambda_n$ be the eigenvalues of $(S^H)^{-1}BS^{-1}$. Then we know that
\begin{align*}
  \det(tI + (1-t)(S^H)^{-1}BS^{-1}) = \prod_{i=1}^n(t + (1-t)\lambda_i)
\end{align*}
and the last equality in part (b) becomes
\begin{align*}
  g(t) &= -\log(\det(tI + (1-t)(S^H)^{-1}BS^{-1})) -\log(\det(A))
  \\
  &= -\log(\prod_{i=1}^n(t + (1-t)\lambda_i)) -\log(\det(A))
  \\
  &= -\sum_{i=1}^n\log(t + (1-t)\lambda_i) -\log(\det(A))
\end{align*}

(d) $g''(t) \geq 0$ for all $t \in [0,1]$. \\
\textit{Subproof:} \\
\begin{align*}
  &g(t) = -\sum_{i=1}^n\log(t + (1-t)\lambda_i) -\log(\det(A))
  \\
  &\implies g'(t) = -\sum_{i=1}^n \frac{1-\lambda_i}{t+(1-t)\lambda_i}
  \\
  &\implies g''(t) = -\sum_{i=1}^n \frac{-(1-\lambda)^2}{(t+(1-t)\lambda_i)^2} = \sum_{i=1}^n \frac{(1-\lambda)^2}{(t+(1-t)\lambda_i)^2} \geq 0
\end{align*}

\textbf{Exercise 7.13} \\
If $f: \mathbb{R}_n \to \mathbb{R}$ is convex and bounded above, then $f$ is constant. \\
\textit{Proof:} \\
On the contrary, suppose $f$ is not constant. WLOG, assume that $f(\vec{x}_1) < f(\vec{x}_3)$ for some $\vec{x}_1, \vec{x}_2 \in \mathbb{R}^n$. Since $f$ is convex, we know that for there exists an $\vec{x}_2$ s.t.
\begin{align*}
  f(\vec{x}_2) = f(\lambda \vec{x}_1 + (1-\lambda)\vec{x}_3) &\leq \lambda f(\vec{x}_1) + (1-\lambda)f(\vec{x}_3)
  \\
  & < f(\vec{x}_3)
\end{align*}
Since $f$ is bounded above by hypothesis, we know that there exists some $M \in \mathbb{R}$ s.t. $f(\vec{x}) \leq M$ for all $\vec{x} \in \mathbb{R}^n$. Letting $f(\vec{x}_2) = M$, we know that $f(\vec{x}_3) > M$, which is a contradiction. Therefore, $f$ must be constant. \\

\textbf{Exercise 7.20} \\
If $f: \mathbb{R}_n \to \mathbb{R}$ is convex and $-f$ is also convex, then $f$ is affine.\\
\textit{Proof:} \\
Let $\vec{x}_1, \vec{x}_2 \in \mathbb{R}^n$.
Since $f$ is convex,
\begin{align*}
  f(\lambda \vec{x}_1 + (1-\lambda)\vec{x}_2) \leq \lambda f(\vec{x}_1) + (1-\lambda)f(\vec{x}_2)
\end{align*}
Since $-f$ is also convex,
\begin{align*}
  &-f(\lambda \vec{x}_1 + (1-\lambda)\vec{x}_2) \leq -\lambda f(\vec{x}_1) - (1-\lambda)f(\vec{x}_2)
  \\
  &\implies f(\lambda \vec{x}_1 + (1-\lambda)\vec{x}_2) \geq \lambda f(\vec{x}_1) + (1-\lambda)f(\vec{x}_2)
\end{align*}
Therefore,
\begin{align*}
  f(\lambda \vec{x}_1 + (1-\lambda)\vec{x}_2) = \lambda f(\vec{x}_1) + (1-\lambda)f(\vec{x}_2)
\end{align*}
We can define another function $g(\vec{x}) = f(\vec{x}) - f(\vec{0})$, which has the property $g(\vec{0})=0$.
\begin{align*}
  g(\lambda \vec{x}_1 + (1-\lambda)\vec{x}_2) &= f((\lambda \vec{x}_1 + (1-\lambda)\vec{x}_2)) - f(\vec{0})
  \\
  &= \lambda f(\vec{x}_1) + (1-\lambda)f(\vec{x}_2) - \lambda f(\vec{0}) - (1 - \lambda) f(\vec{0})
  \\
  &= \lambda (f(\vec{x}_1) - f(\vec{0})) + (1-\lambda)(f(\vec{x}_2) - f(\vec{0}))
  \\
  &= \lambda g(\vec{x}_1) + (1-\lambda)g(\vec{x}_2)
\end{align*}
To show that $f$ is affine, it suffices to show that $g$ is linear. It is obvious that $g(\lambda \vec{x}) = \lambda g(\vec{x})$ for $\lambda \in [0, 1]$. For $\lambda > 1$, note that $\frac{1}{\lambda} \in (0, 1)$.
\begin{align*}
  g(\vec{x})
  &= g\Big(\lambda \frac{1}{\lambda} \vec{x} + \Big(1-\frac{1}{\lambda}\Big)(\vec{0})\Big)
  \\
  &= \frac{1}{\lambda} g(\lambda \vec{x}) + (1-\frac{1}{\lambda})g(\vec{0})
  \\
  &= \frac{1}{\lambda} g(\lambda \vec{x})
\end{align*}
\begin{align*}
  \implies \lambda g(\vec{x}) = g(\lambda \vec{x})
\end{align*}
We use the above equality to show that $g(\vec{x}_1 + \vec{x}_2) = g(\vec{x}_1) + g(\vec{x}_2)$.
\begin{align*}
  g(\vec{x}_1 + \vec{x}_2)
  &= g\Big(\frac{1}{2}(2\vec{x}_1 + \frac{1}{2}(2\vec{x}_2)\Big) \\
  &= \frac{1}{2}g(2\vec{x}_1) + \frac{1}{2}g(2\vec{x}_2) \\
  &= g(\vec{x}_1) + g(\vec{x}_2)
\end{align*}
This shows that $g$ is linear $\implies$ $f$ is affine. \\

\textbf{Exercise 7.21} \\
If $D \subset \mathbb{R}$ with $f: \mathbb{R}_n \to D$, and if $\phi: D \to \mathbb{R}$ is a strictly increasing function, then $\vec{x}^*$ is a local minimizer for the problem
\begin{align*}
  &\text{minimize} \qquad \ \phi \circ f(\vec{x}) \\
  &\text{subject to} \qquad G(\vec{x}) \preceq \vec{0} \\
  & \ \qquad\qquad\qquad H(\vec{x}) = \vec{0}
\end{align*}
if and only if $\vec{x}^*$ is a local minimizer for the problem
\begin{align*}
  &\text{minimize} \qquad \ f(\vec{x}) \\
  &\text{subject to} \qquad G(\vec{x}) \preceq \vec{0} \\
  & \ \qquad\qquad\qquad H(\vec{x}) = \vec{0}
\end{align*}
\textit{Proof:} \\
Let $\vec{x}^*$ be a minimizer of $f$. Then $f(\vec{x}^*) \leq f(\vec{x})$ for all $\vec{x}$ in an open neighborhood U around $\vec{x}^*$. Since $\phi$ is strictly increasing, we know that $\phi(f(\vec{x}^*)) \leq \phi(f(\vec{x}))$ for all $\vec{x} \in U$. Therefore $\vec{x}^*$ is a minimizer of $\phi \circ f$.

Now let $\vec{x}^*$ be a minimizer of $\phi \circ f$. Then $\phi(f(\vec{x}^*)) \leq \phi(f(\vec{x}))$ for all $\vec{x}$ in an open neighborhood V. Since $\phi$ is strictly increasing, we know that $f(\vec{x}^*) \leq f(\vec{x})$ for all $\vec{x} \in V$. Therefore $\vec{x}^*$ is a minimizer of $f$.

\end{document}
